\documentclass[article,aoas,preprint]{imsart}

\usepackage[nofiglist, nomarkers]{endfloat}
\usepackage{algorithm}
\usepackage{graphicx}
\usepackage{amsmath}
\usepackage{amssymb}
\usepackage{amsfonts}
\usepackage{amsthm}
\usepackage{xfrac}
\usepackage{float}
\usepackage{fullpage}
\RequirePackage[colorlinks,citecolor=blue,urlcolor=blue]{hyperref}

% override imsart settings for final project
\startlocaldefs
\setattribute{title}{size} {\fontseries{bx}\fontsize{14}{16}\selectfont\mathversion{bold}\spaceskip.5em}
\setattribute{journal}{name}{PH240F: Statistical Genomics II}
\setattribute{author}{prefix}{}
\setattribute{volume}{title}{Spring 2014---Final Project}

\makeatletter
\let\@fnsymbol\@arabic
\makeatother
\numberwithin{equation}{section}
\theoremstyle{plain}
\newtheorem{thm}{Theorem}[section]
\newtheorem{lemma}{Lemma}[section]
\newtheorem{definition}{Definition}
\newtheorem{Rule}{Rule}
\newtheorem*{notation}{Notation}
\endlocaldefs

% my standard setup 
\newcommand{\der}[2]{\frac{d #1}{d #2}} % for derivatives
\newcommand{\V}[1]{\ensuremath{\mathbf{#1}}} % for vectors
\newcommand{\gv}[1]{\ensuremath{\mbox{\boldmath$ #1 $}}} % for vectors of Greek letters
\newcommand{\pd}[2]{\frac{\partial #1}{\partial #2}}  % for partial derivatives
\newcommand{\grad}[1]{\gv{\nabla} #1} % for gradient
\newcommand{\reals}{\mathbb{R}}
\newcommand{\ints}{\mathbb{Z}}
\newcommand{\blank}{\underline{\hspace*{1in}}}
\newcommand{\PMF}{\mathrm{PMF}}
\newcommand{\PDF}{\mathrm{PDF}}
\newcommand{\CDF}{\mathrm{CDF}}
\newcommand{\N}[2]{\mathcal{N}\left(#1,#2\right)}
\newcommand{\empavg}[2]{\frac{1}{#1}\sum_{i=1}^{#1}\left[#2\right]}
\newcommand{\E}[1]{{\rm I\kern-.3em E}\left[#1\right]}
\newcommand{\Var}[1]{\mathrm{Var}\left[#1\right]}
\newcommand{\Cov}[1]{\mathrm{Cov}\left[#1\right]}
\def\ci{\perp\!\!\!\perp}
\newcommand{\argmax}[1]{\underset{#1}{\operatorname{argmax}}}
\newcommand{\argmin}[1]{\underset{#1}{\operatorname{argmin}}}
\newcommand{\iid}{\stackrel{\mathrm{iid}}{\sim}}
\newcommand{\logit}[1]{\operatorname{logit}({#1})}
\providecommand{\e}[1]{\ensuremath{\times 10^{#1}}}
\newcommand{\Tr}[1]{\mathrm{Tr}\left(#1\right)}
\newcommand{\adim}[2]{\underset{\scriptscriptstyle #1}{#2\strut}}

\newcommand{\fix}[1] { \textcolor{red} {
{\fbox{ {\bf Fix:} \ensuremath{\blacktriangleright }} {\bf #1}
\fbox{\ensuremath{\blacktriangleleft} } } } }



\begin{document}

\begin{frontmatter}

\title{Classification and Prediction for Renal Cell Carcinomas}
\runtitle{Renal cell carcinomas classification}

\begin{aug}
\author{\fnms{Alex} \snm{Anderson},\thanksref{t1}\ead[label=e1]{aga@berkeley.edu}}
\author{\fnms{K. Jarrod} \snm{Millman},\thanksref{t2}\ead[label=e2]{millman@berkeley.edu}}
\and
\author{\fnms{Lara} \snm{Troszak}\thanksref{t2}\ead[label=e3]{troszak1@berkeley.edu}}
\thankstext{t1}{Department of Physics, UC Berkeley}
\thankstext{t2}{Division of Biostatistics, School of Public Health, UC Berkeley}
\runauthor{Anderson, Millman, and Troszak}
\end{aug}



\begin{abstract}

In this project, we will compare classification methods for deciphering between
two different cancer cells. Specifically, we will be looking at Kidney renal
papillary cell carcinoma (KIRP) and Kidney renal clear cell carcinoma (KIRC)
from the Cancer Genome
Atlas.\footnote{\url{https://tcga-data.nci.nih.gov/tcga/}}

After downloading the raw RNAseq count data from the website, we will perform
exploratory data analysis, looking at the need for normalization, low count
filtering, batch effects, and outlier removal.

Moving forward, we will split the data into training (2/3) and validation sets
(1/3) for each cancer cell type. Then, using the training sets we will rank the
genes in terms of differential expression between cell types using a naive
t-test. These ranks will be used for feature selection (top 10, 20, 30, 40
genes) to compare classification performance based on different numbers of
features.

Finally, we will perform several classification methods per feature set
(training our methods on the training data sets and testing our methods on the
validation data sets): K-Nearest Neighbors, Linear Discriminant Analysis, and
Support Vector Machines.  Additionally, we will compare these methods to the
Random Forest classifier.

 
\end{abstract}

\begin{keyword}
\kwd{Renal Cell Carcinomas}
\kwd{Classification}
\kwd{Prediction}
\end{keyword}

\end{frontmatter}

\section*{Acknowledgments}
We would like to thank Davide Risso for pointing us to the Cancer Genome
Atlas as well as providing feedback on our proposal.

\end{document}
