\documentclass[article,aoas,preprint]{imsart}

\usepackage[nofiglist, nomarkers]{endfloat}
\usepackage{algorithm}
\usepackage{graphicx}
\usepackage{amsmath}
\usepackage{amssymb}
\usepackage{amsfonts}
\usepackage{amsthm}
\usepackage{xfrac}
\usepackage{float}
\usepackage{fullpage}
\RequirePackage[colorlinks,citecolor=blue,urlcolor=blue]{hyperref}

% override imsart settings for final project
\startlocaldefs
\setattribute{title}{size} {\fontseries{bx}\fontsize{14}{16}\selectfont\mathversion{bold}\spaceskip.5em}
\setattribute{journal}{name}{PH240F: Statistical Genomics II}
\setattribute{author}{prefix}{}
\setattribute{volume}{title}{Spring 2014---Final Project}

\makeatletter
\let\@fnsymbol\@arabic
\makeatother
\numberwithin{equation}{section}
\theoremstyle{plain}
\newtheorem{thm}{Theorem}[section]
\newtheorem{lemma}{Lemma}[section]
\newtheorem{definition}{Definition}
\newtheorem{Rule}{Rule}
\newtheorem*{notation}{Notation}
\endlocaldefs

% my standard setup 
\newcommand{\der}[2]{\frac{d #1}{d #2}} % for derivatives
\newcommand{\V}[1]{\ensuremath{\mathbf{#1}}} % for vectors
\newcommand{\gv}[1]{\ensuremath{\mbox{\boldmath$ #1 $}}} % for vectors of Greek letters
\newcommand{\pd}[2]{\frac{\partial #1}{\partial #2}}  % for partial derivatives
\newcommand{\grad}[1]{\gv{\nabla} #1} % for gradient
\newcommand{\reals}{\mathbb{R}}
\newcommand{\ints}{\mathbb{Z}}
\newcommand{\blank}{\underline{\hspace*{1in}}}
\newcommand{\PMF}{\mathrm{PMF}}
\newcommand{\PDF}{\mathrm{PDF}}
\newcommand{\CDF}{\mathrm{CDF}}
\newcommand{\N}[2]{\mathcal{N}\left(#1,#2\right)}
\newcommand{\empavg}[2]{\frac{1}{#1}\sum_{i=1}^{#1}\left[#2\right]}
\newcommand{\E}[1]{{\rm I\kern-.3em E}\left[#1\right]}
\newcommand{\Var}[1]{\mathrm{Var}\left[#1\right]}
\newcommand{\Cov}[1]{\mathrm{Cov}\left[#1\right]}
\def\ci{\perp\!\!\!\perp}
\newcommand{\argmax}[1]{\underset{#1}{\operatorname{argmax}}}
\newcommand{\argmin}[1]{\underset{#1}{\operatorname{argmin}}}
\newcommand{\iid}{\stackrel{\mathrm{iid}}{\sim}}
\newcommand{\logit}[1]{\operatorname{logit}({#1})}
\providecommand{\e}[1]{\ensuremath{\times 10^{#1}}}
\newcommand{\Tr}[1]{\mathrm{Tr}\left(#1\right)}
\newcommand{\adim}[2]{\underset{\scriptscriptstyle #1}{#2\strut}}

\newcommand{\fix}[1] { \textcolor{red} {
{\fbox{ {\bf Fix:} \ensuremath{\blacktriangleright }} {\bf #1}
\fbox{\ensuremath{\blacktriangleleft} } } } }



\begin{document}

\begin{frontmatter}

\title{Classification for Renal Cell Carcinomas}
\runtitle{Renal cell carcinomas classification}

\begin{aug}
\author{\fnms{Alex} \snm{Anderson},\thanksref{t1}\ead[label=e1]{aga@berkeley.edu}}
\author{\fnms{K. Jarrod} \snm{Millman},\thanksref{t2}\ead[label=e2]{millman@berkeley.edu}}
\and
\author{\fnms{Lara} \snm{Troszak}\thanksref{t2}\ead[label=e3]{troszak1@berkeley.edu}}
\thankstext{t1}{Department of Physics, UC Berkeley}
\thankstext{t2}{Division of Biostatistics, School of Public Health, UC Berkeley}
\runauthor{Anderson, Millman, and Troszak}
\end{aug}



\begin{abstract}

In this project, we will compare classification methods for discriminating between
two different cancer cells. Specifically, we will be looking at Kidney renal
papillary cell carcinoma (KIRP) and Kidney renal clear cell carcinoma (KIRC)
from the Cancer Genome Atlas.\footnote{\url{https://tcga-data.nci.nih.gov/tcga/}}

Using the unnormalized RNAseq count data from this website, we will perform exploratory
data analysis, looking at the need for normalization, low count filtering,
batch effects, and outlier removal.

Moving forward, we will utilize k-fold cross validation to assess
our classification methods. Using the training sets we will rank the
genes in terms of differential expression between cell types based on their p-values.
These ranks will be used for feature selection, comparing classification performance
based on different numbers of features (selection of the top 10, 20, 30, and 40 
genes as features).

Finally, we will perform several classification methods per feature set
(training our methods on the training data sets and testing our methods on the
validation data sets): K-Nearest Neighbors, Linear Discriminant Analysis, and
Support Vector Machines.  Additionally, we will compare these methods to the
Random Forest classifier.

 
\end{abstract}

\begin{keyword}
\kwd{Renal Cell Carcinomas}
\kwd{Classification}
\kwd{Prediction}
\end{keyword}

\end{frontmatter}



\section{Possible Extensions}

Depending on the amount of time that we have after we finish our core proposal,
we may pursue some of the following refinements of our analysis:

\begin{enumerate}

\item There are many different methods to normalize our data. We could try
these normalizations and see how they impact the final results of our
classifier. 

\item There are many potential ways to choose features to supply to our list of
classification algorithms. 

\item In our current design, we use between sample normalization for a test set
that is one-third of our total sample size. Ideally, we would be able to train
a classifier and feed it just one new RNAseq data set and assess the existence
of cancer. 

\item It would be helpful for our classifier to have the following three
outputs: no cancer, KIRP, KIRC. This would require the collection and
formatting of additional data from healthy kidney cells. 

\end{enumerate}

\section*{Acknowledgments}
We would like to thank Davide Risso for pointing us to the Cancer Genome
Atlas as well as providing feedback on our proposal.

\end{document}
